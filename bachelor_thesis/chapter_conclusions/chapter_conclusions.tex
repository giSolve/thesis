\chapter{Conclusions}\label{chapter:conclusions}
The goal of this work was to find a method to compute stochastic quantities of Dirichlet Laplacian eigenvalue problems.
We proposed an approach based on computing expectations of eigenspaces rather than of the individual eigenpairs themselves, which means that we completely avoid the matching problem that arises for the latter quantities.
For this purpose, we required discretization methods to solve the Dirichlet Laplacian eigenvalue problem which retain the structure of the problem.
We have seen that this is the case for the Galerkin approximation of the single layer boundary integral operator as well as for the contour integral method.
Moreover, the contour integral method was a particularly suitable method to solve the nonlinear eigenvalue problem arising from the Galerkin discretization as it allows for a simultaneous computation of various eigenvalues and their eigenspaces.

Our numerical experiments seem to confirm the theoretical results.
So, we have found that we can satisfactorily evaluate stochastic properties of our eigenvalue problem given that we can locate and separate the eigenvalues of interest in and by a contour.
Employing a multilevel quadrature accelerates our method.
Still, the most expensive part of the algorithm, which is markedly computing the approximation of the single layer boundary integral operator, prevents us from computing reference solutions with more samples.