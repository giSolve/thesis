\documentclass[paper=a4, fontsize=11pt, BCOR=13mm, DIV=13, headinclude, toc=index, toc=bibliography, english, twoside, parskip]{scrreprt}
% Die verwendete Dokumentenklasse ist scrreprt. Die verwendeten Optionen sind:
%
% paper=a4              Papier ist a4.
% fontsize=11pt         Schrifgröße ist 11.
% DIV=13                Das Papier wird in d viele Spalten und d' viele Zeilen eingeteilt. Die Werte werden aus DIV berechnet.
% BCOR=1cm              Definiert den Patz, der auf der Innenseite beim Binden verloren geht.
% headinclude           sorgt dafür, dass genug Platz für die Header vorhanden ist.
% toc=index             legt im Inhaltsverzeichnis einen Eintrag für das Stichwortverzeichnis an.
% toc=bibliography      legt im Inhaltsverzeichnis einen Eintrag für das Literaturverzeichnis an.
% english               englische Worte wie "Chapter" und "References".
% twoside               Beidseitiges Dokument, wie in einem Buch.


%If you don't want this fancyheaders comment out lines 17 to 24
% Fancyheader
\usepackage{fancyhdr}                   % Wie der Name schon sagt, um fancy Header zu generieren.
\pagestyle{fancy}                       % Fancy Header sollen angezeigt werden
\renewcommand{\sectionmark}[1]{\markright{\thesection.\ #1}{}}    % Verhindert dass rightmark ausschließlich Grußbuchstaben benutzt
\fancyhead[LE,RO]{\rightmark}           % Links bei geraden und rechts bei ungeraden Seitenzahlen soll der Name der Section stehen.
\fancyhead[LO,RE]{}                     % Links bei ungeraden und rechts bei geraden Seitenzahlen soll nichts stehen.
\fancyfoot[C]{}                         % Keine mittigen Seitenzahlen
\fancyfoot[LE,RO]{\thepage}             % Seitenzahlen unten in die jeweilige äußere Ecke





\setcounter{secnumdepth}{3}     % Nummerierungstiefe (chapter, section, subsection, ...).
\setcounter{tocdepth}{3}        % Nummerierungstiefe im Inhaltsverzeichn is.

\usepackage[linesnumbered,ruled,vlined]{algorithm2e}    % Algorithmen setzen.
\newcommand{\SkipBeforeAndAfter}{\vspace{8mm}}
%\SetAlgoSkip{SkipBeforeAndAfter}
%\SetAlgoSkip{bigskip} 
%\usepackage{algorithm}
%\usepackage{algpseudocodex}
%\algrenewcommand\algorithmicrequire{\textbf{Input:}}
%\algrenewcommand\algorithmicensure{\textbf{Output:}}
%\algnewcommand{\LeftComment}[1]{\Statex \(\triangleright\) #1}
\usepackage{amsmath,amssymb,amsthm,amsfonts,amsbsy,latexsym}    % "Notwendige" AMS-Math Pakete.
\usepackage{array}                      % Bessere Tabellen.
\renewcommand{\arraystretch}{1.15}      % Tabellen bekommen ein wenig mehr Platz.
\usepackage{bbm}                        % Dicke 1.
\usepackage[utf8]{inputenc}             % utf8 als Eingabeformat. should be loaded before biblatex
\usepackage[backend=biber, style=alphabetic, giveninits=true]{biblatex}  % Gute Erweiterung zu bibtex, Wird für Referenzen benutzt.
%\bibliography{masterthesis_your_name_bibliography}   % Die verwendeten Referenzen (.bib-Datei)
\addbibresource{masterthesis_your_name_bibliography.bib}
\usepackage[hypcap]{caption}            % Damit Hyperrefs bei der figure-Umgebung auf die Figure zeigt statt auf die Caption.
\captionsetup{font=footnotesize} 
\usepackage{hyperref}
\usepackage{cleveref}

\usepackage{datetime}                   % Um \today einzustellen.
\newdateformat{mydate}{\THEDAY{}th \monthname{} \THEYEAR{}}
\usepackage{diagbox}                    % Diagonale in Tabellen.
\usepackage{enumitem}                   % Zum Ändern der Nummerierungsumgenung 'enumerate'
\setlist[enumerate,1]{label=(\roman*)}  % Aufzählungen sind vom Typ 'Klammer auf; kleine römische Zahl; Klammer zu'
\usepackage[T1]{fontenc}                % Bessere Schrift
\usepackage{ifthen}                     % Zum checken ob Parameter leer sind.
\usepackage{lmodern}                    % Bessere Schrift
\usepackage{listings}                   % Code Listings.
\usepackage{mathtools}                  % Subscript unter Summen behandeln. Der Befehl lautet \mathclap.
\usepackage{makeidx}                    % Stichwortverzeichnis.
\makeindex                              % Stichwortverzeichnis erstellen.
\renewcommand{\indexname}{Index}        % Name des Index definieren.
\usepackage{multirow}                   % In Tabellen mehrere Zeilen zu einer machen.
%\usepackage[parfill]{parskip}
\usepackage{rotating}                   % Um Figures zu drehen.
\usepackage{scrhack}                    % Verbessert die Zusammenarbeit von KOMA mit anderen Paketen (z.B, listing).
\usepackage{stackrel}                   % Symbole übereinander stapeln.
\usepackage[dvipsnames]{xcolor}         % Gefärbter Text und so.
\usepackage{tikz}                       % Graphen und kommutative Diagramme. Muss nach xcolor eingebunden werden.
\usepackage{tikz-cd}                    % Kommutative Diagramme.
\usepackage{transparent}                % Braucht mal manchmal für inkscape bilder.
\usepackage{csquotes}                   % für \enquote
\usetikzlibrary{patterns}               % Zu malen von schraffierten Flächen.

\graphicspath{{pictures/}}              % Pfad in dem die mit Inkscape erstellen Bilder liegen (relativ zum Hauptverzeichnis).

% Workaround, damit keine unnötigen Leerzeichen entstehen.
\let\oldindexdefn\index
\renewcommand*{\index}[1]{\oldindexdefn{#1}\ignorespaces}
\let\oldlabeldefn\label
\renewcommand*{\label}[1]{\oldlabeldefn{#1}\ignorespaces}

% Workaround, Linebreak nach ldots erlaubt.
\newcommand{\origldots}{}
\let\origldots\ldots
\renewcommand{\ldots}{\allowbreak\origldots}


% Symbolverzeichnis
\usepackage[intoc, english]{nomencl}     % Symbolverzeichnis.
% intoc                 die Symbolliste in das Inhaltsverzeichnis aufnehmen.
% english               englische Worte wie "Seite".
\renewcommand{\nomname}{Symbol Index}   % Definiert die Überschrift des Symbolverzeichnises.
\renewcommand{\nomlabelwidth}{80pt}     % Platz der einem Symbol gegönnt wird.
\newcommand{\symbolindex}[4][]{{\nomenclature[#1]{#2}{#3\ifthenelse{\equal{#4}{}}{}{ -- #4}\nomnorefpage}}\ignorespaces}    % Verbesserte Version von "\nomenclature". Erzeugt Symbol Beschreibung - Referenz.
\renewcommand*{\nompreamble}{\markright{\nomname}}    % Workaround: Fancyhdr schreibt im Symbolverzeichnis sonst den Namen des leztztes Kapitels.
\makenomenclature                       % Symbolverzeichnis erstellen.

% Anklickbare Referenzen (letztes eingebundenes Paket)
\usepackage{hyperref}                   % Referenzen innerhalb des Dokuments anklickbar machen. Achtung: Muss das letzte Paket im Präambel sein.
\hypersetup{                            % Optionen von hyperref Einstellen.
    colorlinks=true,                    % gefärbte Links an Stelle von Boxen.
    linkcolor=black,                     % Farbe interner Links.
    citecolor=black,                     % Farbe von Referenzen.
    urlcolor=black                       % Farbe von Internetlinks.
}

\usepackage{BA_Titelseite}


%Namen des Verfassers der Arbeit
\author{Solveig Tr\"ankner}
%Geburtsdatum des Verfassers
\geburtsdatum{12. März 2002}
%Gebortsort des Verfassers
\geburtsort{Wiesbaden, Hessen}
%Datum der Abgabe der Arbeit
%\date{\today}
\date{21. M\"arz 2025}

%Name des Betreuers
% z.B.: Prof. Dr. Peter Koepke
\betreuer{Betreuer: Prof. Dr. Jochen Garcke}
%Name des Zweitgutachters
\zweitgutachter{Zweitgutachter: Dr. Bastian Bohn}
%Name des Instituts an dem der Betreuer der Arbeit tätig ist.
%z.B.: Mathematisches Institut
%\institut{Institut XYZ}
%\institut{Mathematisches Institut}
%\institut{Institut f\"ur Angewandte Mathematik}
\institut{Institut f\"ur Numerische Simulation}
%\institut{Forschungsinstitut f\"ur Diskrete Mathematik}
%Titel der Bachelorarbeit
\title{Data Visualization with t-SNE in Theory and Practice}
%Do not change!
\ausarbeitungstyp{Bachelorarbeit Mathematik}


%       Theoreme
\theoremstyle{definition}               % Name: dick            Text: normal.
%\newtheorem{defi}{Definition}[section]  % Der Zähler ist defi = Sectionzähler.1 . Sectionzähler soll bei Benutzung von defi nicht erhöht werden.
\newtheorem{defi}{Definition}[chapter]
\newtheorem*{defi*}{Definition}
\newtheorem{example}[defi]{Example}
\newtheorem{notation}[defi]{Notation}
\newtheorem{rem}[defi]{Remark}
\AtBeginEnvironment{rem}{%
  \pushQED{\qed}\renewcommand{\qedsymbol}{$\triangle$}%
}
\AtEndEnvironment{rem}{\popQED\endexample}
\newtheorem{defcor}[defi]{Definition/Corollary}
\newtheorem{defprop}[defi]{Definition/Proposition}
\newtheorem{defthm}[defi]{Definition/Theorem}

\newtheorem*{idea}{Idea}
\newtheorem*{question}{Question}
\newtheorem*{obs}{Observation}
\newtheorem*{prob}{Problem}
\newtheorem*{but}{But}
\newtheorem*{reci}{Recipe}

\theoremstyle{plain}
\newtheorem*{conj}{Conjecture}
\newtheorem{cor}[defi]{Corollary}
\newtheorem{lem}[defi]{Lemma}
\newtheorem{prop}[defi]{Proposition}
\newtheorem*{prop*}{Proposition}
\newtheorem{thm}[defi]{Theorem}
\newtheorem*{thm*}{Theorem}


%       Makros
\newcommand{\bA}{\mathbb{A}}
\newcommand{\bB}{\mathbb{B}}
\newcommand{\bC}{\mathbb{C}}
\newcommand{\bD}{\mathbb{D}}
\newcommand{\bE}{\mathbb{E}}
\newcommand{\bF}{\mathbb{F}}
\newcommand{\bG}{\mathbb{G}}
\newcommand{\bH}{\mathbb{H}}
\newcommand{\bI}{\mathbb{I}}
\newcommand{\bJ}{\mathbb{J}}
\newcommand{\bK}{\mathbb{K}}
\newcommand{\bL}{\mathbb{L}}
\newcommand{\bM}{\mathbb{M}}
\newcommand{\bN}{\mathbb{N}}
\newcommand{\bO}{\mathbb{O}}
\newcommand{\bP}{\mathbb{P}}
\newcommand{\bQ}{\mathbb{Q}}
\newcommand{\bR}{\mathbb{R}}
\newcommand{\bS}{\mathbb{S}}
\newcommand{\bT}{\mathbb{T}}
\newcommand{\bU}{\mathbb{U}}
\newcommand{\bV}{\mathbb{V}}
\newcommand{\bW}{\mathbb{W}}
\newcommand{\bX}{\mathbb{X}}
\newcommand{\bY}{\mathbb{Y}}
\newcommand{\bZ}{\mathbb{Z}}

\newcommand{\bfn}{\mathbf{n}}
\newcommand{\bfx}{\mathbf{x}}
\newcommand{\bfy}{\mathbf{y}}
\newcommand{\bs}[1]{{\boldsymbol#1}}

%\newcommand{\dd}{\mathrm{d}} 
\newcommand{\dd}{\operatorname{d}\!}
\newcommand{\inv}{^{-1}}
\newcommand*{\hm}{^H}
\newcommand*{\tp}{^T}

\newcommand\cconj[1]{\mathop{\overline{#1}}}
\newcommand\closure[1]{\overline{#1}}

\DeclareMathOperator{\spn}{span}
\DeclareMathOperator{\intr}{\text{int}}
\DeclareMathOperator{\rk}{\text{rank}}
\DeclareMathOperator{\diag}{diag}
\DeclareMathOperator{\Ima}{\text{Im}}

%\DeclarePairedDelimiter{\abs}{\lvert}{\rvert}
%\DeclarePairedDelimiter{\norm}{\lVert}{\rVert}

\newcommand\comment[1]{\textcolor{magenta}{\emph{#1}}}


\begin{document}
% Titelseite
\maketitle              % Titelseite ausgeben

\begin{abstract}
    \textbf{Abstract (English)}

    The t-SNE algorithm has been used extensively for visualizing high-dimensional data in multiple areas, from single-cell biology to natural language processing and computer vision and engineering. 
    We describe the t-SNE algorithm and present a summary of theoretical research results on it. 
    In particular, we present and critically analyze a theorem that provides a clustering guarantee for t-SNE and investigate the proposal of a rescaled t-SNE that is consistent in the large data limit. 
    We give an overview of t-SNE from the practical point of view, outlining the algorithmic aspects that make the algorithm faster and present an automated stopping strategy. 
    We also summarize the consensus on choosing the best hyperparameters. 
    We investigate multiple different parameters through experiments.
    \textcolor{red}{Say something about results here.}

    \vspace{2em}
    \textbf{Abstract (German)}

    In den letzten Jahren wurde der t-SNE-Algorithmus in zahlreichen Bereichen zur Visualisierung hochdimensionaler Daten eingesetzt, von der Zellbiologie über die Verarbeitung natürlicher Sprache bis hin zu Computer Vision und Engineering. 
    Wir beschreiben den t-SNE-Algorithmus und präsentieren eine Zusammenfassung der theoretischen Forschungs\-ergebnisse zu diesem Verfahren. 
    Insbesondere stellen wir ein Theorem vor und analysieren es kritisch, das eine Clustering-Garantie für t-SNE bietet, und untersuchen den Vorschlag eines neu skalierten t-SNE, der im Grenzwert großer Datenmengen konsistent ist. 
    Wir geben einen Überblick über t-SNE aus praktischer Sicht, skizzieren die algorithmischen Aspekte, die den Algorithmus schneller machen, und stellen eine automatische Stoppstrategie vor. 
    Wir fassen auch den Konsens über die Wahl der besten Hyperparameter zusammen. 
    Wir untersuchen mehrere verschiedene Parameter durch Experimente.
    \textcolor{red}{Hier Resultate zusammenfassen.}

\end{abstract}
\cleardoublepage

\setcounter{page}{1}    % Die Titelseite und die darauffolgende leere Seite sollen gefälligst Seite 1 und 2 sein.
\tableofcontents        % Inhaltsverzeichnis ausgeben
%\listofalgorithms

% Einleitung
\cleardoublepage        % Kapitel immer rechts beginnen
\chapter{Introduction}
Notes from \cite{vdMaa08}: 
\begin{itemize}
    \item t-SNE is capable of capturing both local and global structure reasonably well 
\end{itemize}
%This insight is used to find a Monte Carlo sampling strategy that stays meaningful for multiple eigenvalues.

% Modelle für den Modulraum
\cleardoublepage        % Kapitel immer rechts beginnen
\chapter{Dimensionality Reduction}\label{chapter:dimensionality-reduction}

Hello Hello! 

- talk a bit about PCA here 
- linear and nonlinear dimensionality reduction 
- why do we need it? 

\cleardoublepage        % Kapitel immer rechts beginnen
\chapter{t-SNE}\label{chapter:t-sne}

The t-SNE algorithm was first proposed in 2008 by Laurens van der Maaten and Geoffrey Hinton \cite{vdMaa08}. 
This built on previous work by Hinton and Roweis, who proposed the precursor \enquote{Stochastic Neighbor Embedding} algorithm in 2002 \cite{Hinton02}. 

\section{Basic t-SNE Algorithm}
This section closely follows the Theoretical Foundations Paper \cite{JMLR:v23:21-0524} and \cite{vdMaa08} so far. 

Let $\{x_1, \dots , x_n \}$ with $x_i \in \mathbb{R}^d$ for all $1 \leq i \leq n$ be the a set of high-dimensional points we wish to visualize.
We initialize a low-dimensional map $\{y_1, \dots , y_n\} \subset \mathbb{R}^2$ (Sometimes, a three-dimensional map is also considered. For the purpose of this thesis, we stick to two-dimensional maps). 

\subsection{Measuring High-Dimensional Similarities}

Instead of using the raw Euclidean distances between the high-dimensional data points, t-SNE measures pairwise similarities via probabilities. 
We define a joint probability distribution over all pairs of data points $\{(x_i, x_j)\}_{1 \leq i \neq j \leq n}$ via  
\begin{equation}
    p_{j|i} =  \frac{\exp(-\norm{x_i - x_j}_2^2 / 2\sigma_i^2)}{\sum_{k \in \{1,2, \dots, n\} \backslash \{i\}} \exp({-\norm{x_i - x_k}_2^2 / 2 \sigma_i^2})}
\end{equation}
which is then symmetrized to $p_{ij} = \frac{p_{i|j} + p_{j|i}}{2n}$. We set $p_{ii}=0$ for all $i$ since we are only interested in modelling pairwise similarities between points. In matrix form, we write $P = (p_{ij})_{1 \leq i, j \leq n}$. 
Large $p_{ij}$ values indicate that the points $x_i$ and $x_j$ closely resemble each other. 
Intuitively, one can think of $p_{j|i}$ as the probability that $x_i$ would choose $x_j$ as a neighbor if neighbors are chosen according to a Gaussian centered at $x_i$ with bandwidth $\sigma_i$. 

The bandwidths $\sigma_i$ of the Gaussian kernel can be adapted based on a fixed number called perplexity using binary search. See more on this in the section on perplexity. 

One might wonder why it is necessary to symmetrize the similarities. Just taking the $p_{j|i}$ leads to problems for outlier datapoints: since all pairwise distances $\norm{x_i - x_j}^2$ are large for an outlier datapoint $x_i$, all $p_{ij}$ values for this datapoint end up being very small. 
As a result, the location of the low-dimensional map point $y_i$ does not contribute very much to the loss function. With symmetrization, we ensure that $\sum_{j} p_{ij} > \frac{1}{2n}$ for all datapoints $x_i$. 

\subsection{Measuring Low-Dimensional Similarities}
In order to measure similarities between points in the low-dimensional embedding, a first approach would be to also use a Gaussian distribution to convert pairwise distances into probabilities, as above. This is the approach taken originally in SNE \cite{Hinton02}.

However, to address the crowding problem, we instead use the more heavy-tailed Student t-distribution with one degree of freedom (also called a Cauchy distribution). That way, moderate distances in the high-dimensional space will be modeled by much larger distances in the low-dimensional map (since we try to minimize the difference between $p_{ij}$ and $q_{ij}$, see details below). 
\textcolor{red}{\textbf{TO DO}: explain this better}


We then also compute a similarity measure for points in the low-dimensional embedding as follows: 
\begin{equation}
    q_{ij} = \frac{(1+ \norm{y_i - y_j}_2^2)^{-1}}{\sum_{k, l \in \{1,2, \dots, n\}, k \neq l} (1+\norm{y_k - y_l}_2^2)^{-1}}
\end{equation}
where we again define $q_{ii} = 0$ for all $1 \leq i \leq n$. We can collect all of the points in a symmetrical matrix $Q = (q_{ij})_{1 \leq i, j \leq n}$. 


\begin{figure}[h]
    \begin{center}
        \includegraphics[width=0.9\linewidth]{Gaussian_Cauchy.png}
    \end{center}
\end{figure}

\subsection{The Loss Function}
The goal of the algorithm is now to get the similarities $P$ and $Q$ to be as close to each other as possible. 
A common choice for measuring the distance between two distributions is the Kullback-Leibler divergence. 

\begin{defi}[Kullback-Leibler Divergence]
    The \emph{Kullback-Leibler divergence} between two probability distributions $P$ and $Q$ over the same probability space is defined as:
    \[
    D_{\text{KL}}(P \parallel Q) = \sum_{x \in \mathcal{X}} P(x) \log\frac{P(x)}{Q(x)}
    \]
    for discrete distributions where \(\mathcal{X}\) is the domain of the distributions.
\end{defi}

The t-SNE algorithm now aims to find a low-dimensional representation $\mathcal{Y} = (y_1, \dots, y_n)$ that minimizes the KL-divergence between the similarity matrices $P$ and $Q$. 
We thus define the following loss function: 
\begin{equation}
    C(\mathcal{Y}) = D_{\text{KL}}(P \parallel Q) = \sum_{i,j \in \{1,\dots,n\}, i \neq j} p_{ij} \log \frac{p_{ij}}{q_{ij}}
\end{equation}
which leads to the following optimization problem: 
\begin{equation}
    (y_1, \dots y_n) = \argmin_{y_1, \dots y_n} C(\mathcal{Y}) = \argmin_{y_1, \dots y_n} \sum_{i,j \in \{1,\dots,n\}, i \neq j} p_{ij} \log \frac{p_{ij}}{q_{ij}}.
\end{equation}


Note that the Kullback-Leibler divergence is in fact not a metric, since it is not symmetric. 
One can observe that a large $p_{ij}$ being modeled by a small $q_{ij}$ leads to a bigger summand than using a large $q_{ij}$ to model a small $p_{ij}$. 
This means that our loss function places a large cost on using far-apart points to model points that are close in the original dataset. 
On the other hand, there is only a small cost to model points that are actually far apart as nearby in the embedding. 
This shows that we can expect a bigger focus on the preservation of local structure and is important to keep in mind when interpreting t-SNE embeddings, see \cite{Wa16Distill}. 

\subsection{Gradient Descent to Minimize Loss}
Minimizing the cost function can be achieved using a standard gradient-descent type algorithm, with an updating equation of 
\begin{equation}
    y_i^{(k+1)} = y_i^{(k)} + h \frac{\partial C}{\partial y_i}^{(k)} + m^{(k+1)}(y_i^{(k)} - y_i^{(k-1)}) 
\end{equation}
for $i=1,\dots,n$, where $h >0$ is a prespecified step size parameter, $m^{(k)} > 0$ is a momentum parameter and we denote the gradient of our loss function (with respect to $y_i$) as: 
%\begin{equation}
  %  \frac{\partial C}{\partial y_i}^{(k)} = 4 \sum_{1 \leq j \leq n, j \neq i} (y_j^{(k)} - y_i^{(k)}) S_{ij}^{(k)} \in \mathbb{R}^2 \text{ with } S_{ij}^{(k)} = \frac{p_{ij} - q_{ij}^{(k)}}{1+ \norm{y_i^{(k)}-y_j^{(k)}}_2^2 } \in \mathbb{R}
%\end{equation}
\begin{equation}
    \frac{\partial C}{\partial y_i}^{(k)} = 4 \sum_{1 \leq j \leq n, j \neq i} (p_{ij} - q_{ij}^{(k)}) q_{ij}^{(k)} Z^{(k)} (y_j^{(k)} - y_i^{(k)})
\end{equation}
where $Z$ is a global normalization constant: 
\begin{equation}
    Z^{(k)} = \sum_{m \neq l} (1 + \norm{y_m^{(k)} - y_l^{(k)}})^{-1}. 
\end{equation}
\textcolor{red}{\textbf{TO DO}: Check the gradient here, I'm pretty sure it should be $(y_i - y_j)$ and not the other way around}


% algorithm pseudocode here
\begin{algorithm}[H]
    \caption{Basic version of t-Distributed Stochastic Neighbor Embedding}
    \label{alg:tsne}
    \KwIn{data set $\mathcal{X} = \{x_1, x_2, \dots, x_n\}$, perplexity $\text{Perp}$, number of iterations $T$, learning rate $\eta$, momentum $\alpha(t)$}
    \KwOut{Low-dimensional representation $\mathcal{Y}^{(T)} = \{y_1, y_2, \dots, y_n\}$}


    Compute $p_{ij}$ with perplexity $\text{Perp}$ \;
    Sample initial solution $\mathcal{Y}^{(0)} = \{y_1, y_2, \dots, y_n\} \sim \mathcal{N}(0, 10^{-4} I)$\

    \For{$t = 1$ \KwTo $T$}{
        Compute low-dimensional affinities $q_{ij}$ \;
        Compute gradient $\frac{\delta C}{\delta \mathcal{Y}}$ \;
        Update solution:\;
        \[
        \mathcal{Y}^{(t)} = \mathcal{Y}^{(t-1)} + \eta \frac{\delta C}{\delta \mathcal{Y}} + \alpha(t) (\mathcal{Y}^{(t-1)} - \mathcal{Y}^{(t-2)})
        \]
    }

    \Return $\mathcal{Y}^{(T)}$\
\end{algorithm}
\textcolor{red}{\textbf{TO DO}: make algorithm look nicer}

To optimize the gradient descent prodcedure, \cite{vdMaa08} proposes the following: 
\begin{itemize}
    \item setting the momentum term to $\alpha^{(t)} = 0.5$ for $t<250$ and $\alpha^{(t)} = 0.8$ for $t \geq 250$
    \item set intial learning rate to $\eta = 100$ and update after every iteration adaptively using the adaptive learning rate scheme described by Jacobs \cite{Jacobs1988}
\end{itemize}



\section{Optimizing t-SNE Parameters and Hyperparameters}

\subsection{Perplexity}
Let us come back to the variance $\sigma_i$ of the Gaussian centered at each datapoint $x_i$. What is a good value to choose? 
If we fixed a single value $\sigma$ to be the same for every datapoint, this is likely not a good choice, because real-life data often does not have a constant density everywhere but instead has sparser and denser regions. 
Given that we want to consider approximately the same number of nearest neighbors for each $x_i$, we opt to choose larger values of $\sigma_i$ for sparse regions and smaller bandwidths for dense regions. 

\textcolor{red}{\textbf{TO DO}: find a better way to describe what perplexity actually does, the following is just taken directly from \cite{vdMaa08}}

Any particular value of $\sigma_i$ induces a probability distribution $P_i$ over all other datapoints. 
The entropy of this distribution increases as $\sigma_i$ increases. 
The user can specify a specific so-called perplexity
\begin{equation}
    \kappa = \text{Perp}(P_i) = 2^{H(P_i)} 
\end{equation}
where $H(P_i) = -\sum_{j} p_{ij} \log_2 p_{ij}$ denotes the Shannon entropy of $P_i$. 

\textcolor{red}{\textbf{TO DO}: do we use $p_{ij}$ or $p_{j|i}$ in the definition of perplexity?}

Then, t-SNE performs a binary search for the value of $\sigma_i$ that produces the user-specified perplexity. 

One can think of perplexity as a smooth measure of the effective \textcolor{red}{\textbf{TO DO}: what does this actually mean?} number of neighbors being considered in the calculation of the $p_{ij}$. As such, larger perplexity values are computationally more expensive. 

An important question is: which values are good perplexity values. Several suggestions have been made \textcolor{red}{\textbf{TO DO}: compile research on suggestions here}. 


\subsection{Initialization}
The standard t-SNE algorithm starts with an initization $y_i^{(0)}$ for $i=\{1, 2, \dots, n\}$ which are drawn independently from $\mathcal{N}(0, \delta^2 I)$ for some small $\delta > 0$, see \cite{vdMaa08} and \cite{vdMaa14}. 

However, recent work \cite{kobak21} has shown that informative initialization leads to embeddings that better preserve large-scale structures within the data. 
They argue that using informative intialization should be the default option for t-SN
E.
Indeed, modern implementations of t-SNE in libraries like openTSNE \cite{openTSNE} or Scikit-learn now all use PCA initialization by default. 
This means that we perform a principal component analysis on the input data $x_1, \dots, x_n$ and use the output $y_1, \dots, y_n$ as the initial points for the low-dimensional embedding. 

\begin{figure}[h]
    \centering 
    \includegraphics[width=\linewidth]{figures/pca_vs_random_init.png}
    \caption{PCA versus random initialization}
    \label{fig:PCA_vs_random}
\end{figure}
In our experiments, we do not necessarily see a visual improvement of the PCA initialized embedding over the random initialization one, but \textcolor{red}{one should look at other metrics to see how well the global structure is preserved e.g. Pearson correlation or Spearman correlation}. 

\subsection{Early Exaggeration}
Early exaggeration was first proposed as a method of optimizing t-SNE in \cite{vdMaa08}. They proposed multiplying all the $p_{ij}$ by a value $\alpha > 0$ for the first few iterations of the algorithm. Since our loss function encourages the $q_{ij}$ to model the $p_{ij}$ as closely as possible, we achieve artificially large $q_{ij}$ values this way. This means that relatively tight clusters are being formed, which can then move around more easily in space, making it easier to find a good global organization of the clusters. 

Open question: What should $\alpha$ be and for how many iterations should we keep EE on? 
\begin{itemize}
    \item \cite{vdMaa08} originally proposed $\alpha = 4$, for $50$ iterations out of $1000$ in total 
\end{itemize}

We can also understand early exaggeration from a dynamical systems viewpoint. 
We start by recalling the gradient of the t-SNE cost function: 
\begin{equation}
    \frac{\partial C}{\partial y_i} = 4 \sum_{j \neq i} (p_{ij} - q_{ij}) q_{ij} Z (y_i - y_j)
\end{equation}
with normalization term $Z = \sum_{k \neq l} (1+ \norm{y_k - y_l}^2 )^{-1}$. Notice that we can split the gradient into two parts 
\begin{equation}
    \frac{\partial C}{\partial y_i} = 4 (F_{\text{attr}} + F_{\text{rep}}) = 4 \left( \sum_{j \neq i } p_{ij} q_{ij} Z (y_i - y_j) - \sum_{j \neq i} q_{ij}^2 Z (y_i - y_j) \right) 
\end{equation}
where $F_{\text{attr}}$ denotes the sum of all attractive forces and $F_{\text{rep}}$ the sum of all repulsive forces. 

This next explanation is taken from \cite{LinStei22}.

Why does it make sense to call these attractive and repulsive forces? Since we want to minimize the cost function, we perform gradient descent and step in the direction of the negative gradient, so we consider the term
\begin{equation}
   - \frac{1}{4} \frac{\partial C}{\partial y_i} = \sum_{j \neq i } p_{ij} q_{ij} Z (y_j - y_i) - \sum_{j \neq i} q_{ij}^2 Z (y_j - y_i).  
\end{equation}
The first term is considered the attractive term, since it moves the point $y_i$ towards a weighted average of the other $y_i$. 
The weights $p_{ij} q_{ij} Z$ are bigger if the two points are close to each other (both in the low- and high-dimensional space). 
The second term has the opposite sign and thus pushes $y_i$ away from a weighted average of the other points. This time, however, the weights only depend on the closeness of points in the low-dimensional space. 
Put together, this means that the attractive term attracts points that are actually meant to be with each other (based on their similarity in the high-dimensional space) and the repulsive term pushes points apart that get too close in the embedding space, regardless of their real similarity. 


\section{Accelerating t-SNE}

The original t-SNE algorithm is not very fast. Its runtime is $\mathcal{O}(N^2)$, which quickly becomes infeasible for datasets with $100,000$ or more points \textcolor{red}{check the exact number} \comment{explain more why this runtime}. In this section, we will give an overview of the two most popular methods proposed to speed up the algorithm. 

\subsection*{Barnes-Hut t-SNE}
This method uses tree-bases algorithms to speed up t-SNE and was proposed by van der Maaten in 2014 \cite{vdMaa14}. 

\begin{itemize}
    \item the gradient of the t-SNE loss function (KLD) has a natural interpretation as an $N$-body system in which all of the $N$ points in the low-dimensional embedding exert forces on each other. We then need to compute the resultant force on each of the points
\end{itemize}

%\title{Approximating Attractive Forces}
Computing the attractive force is not too expensive computationally, if we approximate input similarities and use vantage-point trees. Recall that input similarities $p_{ij}$ are computed based on a Gaussian kernel. 
Thus, $p_{ij}$ values corresponding to dissimilar input objects $x_i$ and $x_j$ are very small. 
So, it makes sense to develop a sparse approximation for the $p_{ij}$. Instead of computing $N^2$ pairwise similarities, we focus on the $\lfloor 3 \kappa \rfloor$ nearest neighbors of each of the $n$ input objects only, where $\kappa$ denotes the perplexity. We denote the nearest neighbor set of $x_i$ by $\mathcal{N}_i$. 

The similarities are thus given by 
\begin{equation}
    p_{j|i} = \begin{cases}
    \frac{\exp(-\norm{x_i - x_j}^2) / 2 \sigma_i^2}{\sum_{k \in \mathcal{N}_i} \exp(-\norm{x_i - x_k}^2) / 2 \sigma_i^2} & \text{ if } j \in \mathcal{N}_i \\
    0  & \text{ otherwise}
    \end{cases}
\end{equation}
which are again symmetrized then. 

We can find the nearest neighbor sets $\mathcal{N}_i$ in $\mathcal{O}(u N \log N)$ time by building a data structure called a vantage-point tree and performing nearest neighbor search with its help. For details, see \cite{vdMaa14}. 

\textcolor{red}{TODO: maybe say more about vantage-point trees here?}

%\title{Approximating Repulsive Forces}
Naively computing the repulsive forces is not a good idea. It would be in $\mathcal{O}(N^2)$. 
Using the Barnes-Hut algorithm however, this can be sped up to $\mathcal{O}(N \log N)$. 
The algorithm relies on the observation that the repulsive forces exerted between two small groups of points are very similar whenever these two groups are relatively far away from each other. 
More precisely, if we consider points $y_i, y_j$ and $y_k$ with $\norm{y_i - y_j} \approx \norm{y_i - y_k} \gg \norm{y_j - y_k}$, then the contributions of $y_j$ and $y_k$ to $F_{\text{rep}, i}$ will be roughly equal. 

To take advantage of this fact, the Barnes-Hut algorithm constructs a data structure known as a quadtree (or for three-dimensional embeddings an octtree). 
Once we have built a quadtree on the current embedding, we traverse it via depth-first search and decide at every node of the quadtree, whether the corresponding cell can be used as a summary for the contributions to $F_{\text{rep}}$ of all points in the cell or if we need to go deeper. 

\textcolor{red}{Maybe insert a picture of a Quadtree here? But I am not sure how much to write about Barnes-Hut really...}

\subsection*{Fast Interpolation-Based t-SNE}
As observed in \cite{KoBe19SingleCell}, the Barnes-Hut implementation of t-SNE, while faster than the originally proposed algorithm, still becomes provibitively slow for datasets with $n \gg 100,000$. 
The FFT accelerated, fast interpolation-based version of t-SNE proposed by Linderman et. al. in 2019 \cite{Lin19} attempts to further speed up the computation of t-SNE embeddings with a special view towards applications in the analysis of single-cell RNA-seq data, which keeps growing and datasets often have north of 1 million datapoints. 

Rough notes from \cite{Lin19}: 
\begin{itemize}
    \item the repulsive force acting on each point can be expressed via sums of the form \begin{equation}
        \phi(y_i) = \sum_{j=1}^N K(y_i, y_j) q_j 
    \end{equation}
    with either $K_1(y,z) = \frac{1}{1+\norm{y-z}^2}$ or $K_2(y,z) = \frac{1}{(1+\norm{y-z}^2)^2}$ (both of these are smooth and translation-invariant kernels). 
    \item key idea: instead of computing all these sums (this would be in $\mathcal{O}(N^2)$), we instead use polynomial interpolation to approximate the kernels $K$ 
    \item interpolation strategy: subdivide the embedding space into a number of intervals (in the 1D case) or squares (in the 2D case)
    \item in each interval, choose a small, fixed number $p$ of equispaced interpolation nodes 
    \item approximate the kernel $K$ by its low-order polynomial interpolant $K_p$ over these nodes 
    \item This replaces the direct kernel evaluation between every pair with an evaluation on a much coarser grid.
\end{itemize}
TLDR: Low-order polynomial interpolation mediates interactions through a fixed number of nodes, FFT exploits Toeplitz structure to accelerate convolution sums, so we get a near linear runtime. 


% algorithm pseudocode here
\begin{algorithm}[H]
    \caption{FFT-accelerated Interpolation-based t-SNE (FIt-SNE)}
    \label{alg:fit-sne}
    \KwIn{embedding points $\{y_i\}_{i=1}^N$, interpolation coefficients $\{q_i\}_{i=1}^N$, number of intervals $N_{\text{int}}$, interpolation points per interval $p$}
    \KwOut{$\varphi(y_i) = \sum_{i=1}^N K(y_i, y_j) q_j$ for $i=1, \dots, N$}


    For every interval $I_l$, form $p$ equispaced nodes $\tilde{y}_{j, l} = 1/(2N_{\text{int}}p) + \frac{j-1 + (l-1)p}{N_{\text{int}}p}$ for $j=1,\dots,p$. 

    \For{$I = 1$ \KwTo $N_{\text{int}}$}{
        Compute the coefficients $w_{m,l}$ given by 

        \[
            w_{m,l} = \sum_{y_i \in I_l} L_{m, \tilde{y}^l}(y_i) q_i, \text{      } m=1,\dots,p
        \]
    }

    Use FFT to compute values of $v_{m,n}$ given by 

        \[
            \begin{pmatrix}
                v_{1,1} & v_{2,1} & \dots & v_{p, N_{\text{int}}}
            \end{pmatrix}^T = \tilde{K} \begin{pmatrix}
                w_{1,1} & w_{2,1} & \dots & w_{p, N_{\text{int}}}
            \end{pmatrix}^T
        \]
    where $\tilde{K}$ is the Toeplitz matrix given by $\tilde{K}_{ij} = K(\tilde{y_i}, \tilde{y_j})$, $i, j = 1, \dots, N_{\text{int}}p$.
    
    \For{$I = 1$ \KwTo $N_{\text{int}}$}{
        Compute $\varphi(y_i)$ at all points $y_i \in I_l$ via 

        \[
            \varphi(y_i) = \sum_{j=1}^p L_{j, \tilde{y}^l}(y_i) v_{j,l}
        \]
    }

\end{algorithm}

\subsection*{Comparing Barnes-Hut and FIt-SNE}
When comparing the two methods, we only see minor differences in the embedding. 
However, the FIt-SNE version is many times faster than Barnes-Hut t-SNE, which is why we will be using FIt-SNE in all of our experiments going forward. 

As pointed out in \cite{openTSNE}, FIt-SNE scales linearly with the number of smaples, but it introduces additional computational overhead for each embedding.
This is often unneccessary for smaller datasets, which is why openTSNE uses Barnes-Hut t-SNE for data sets with fewer than $10,000$ samples and FIt-SNE for data sets with at least $10,000$ datapoints. 
%\begin{figure}[h]
  %  \centering 
  %  \includegraphics[width=\linewidth]{figures/BH_vs_FFT.png}
 %   \caption{Barnes-Hut versus FIt-SNE on the flow18 dataset}
 %   \label{fig:BH_vs_FIt-SNE}
%\end{figure}




\cleardoublepage        % Kapitel immer rechts beginnen
\chapter{Theoretical Results}\label{chapter:theoretical_results}
\textcolor{red}{TODO: maybe say something about the importance of theoretical work - can lead to better understanding and new directions for improving the algorithm}

t-SNE has been used a lot in practice and there is comparatively less theoretical material available. 
In this chapter, we will first give an overview of the theoretical work on t-SNE that has been done in the last couple of years and then look at a couple of results of interest in more detail. 

\section{Literature Review}
This section is inspired by \cite{murray2024largedatalimitsscaling}. 

Early theoretical work has been focused on establishing clustering guarantees for t-SNE. 
\textcolor{red}{To the best of my knowledge}, Shaham and Steinerberger started this line of investigation with their 2017 paper \enquote{Stochastic Neighbor Embedding Separates Well-Separated Clusters} \cite{shaham17}. 
They proved a clustering result for the precursor algorithm SNE, but their result has been criticised since it is only nontrivial when the number of clusters is significantly larger than the number of points per clusters, which is an unrealistic assumption for most datasets t-SNE is commonly used on, see \cite{Arora18}.  
Linderman and Steinerberger \cite{LinStei22} use a dynamical systems approach for understanding t-SNE. 
They understand the embedding points as particles experiencing attractive and replusive forces and formulate a shrinkage result for the diameter of embedded clusters based on the EE exaggeration factor and gradient descent step size. \\
Building on the preprint of \cite{LinStei22}, \cite{Arora18} point out that Linderman's result does not rule out the possibility of multiple clusters merging into one. 
\cite{Arora18} gives a first formal framework for the problem of data visualization and formulates an improved clustering guarantee for some definition of spherical and well-separated clusterable data. 
However, both the clustering guarantees in \cite{Arora18} and \cite{LinStei22} have been criticised for making assumptions that do not hold in practice, see \cite{yang2021tsneoptimizedrevealclusters}. 


Studying t-SNE in terms of attraction-repulsion dynamics has proven to be one of the most fruitful directions for obtaining theoretical results on t-SNE. 
Steinerberger and Zhang explore coloring of t-SNE plots by the direction of forces acting on each point as a means of obtaining additional information on the embedding \cite{SteiZhang22}. \textcolor{red}{Does this belong here actually?}

In \cite{Cai22}, an asymptotic equivalence of the EE phase with power iterations in spectral embeddings is established. 
This means that for strongly clustered data, one can replace the EE phase with a spectral embedding, thereby speeding up the process. 
They also investigate the embedding phase and characterize it into an amplification and a stabilization phase. 
Practical implications of their work include: 
\begin{itemize}
    \item stopping EE early for noisy data to avoid overshooting (they suggest $K_0 = \lfloor (\log n)^2 \rfloor$ iterations)
    \item the observation that t-SNE is reliable in terms of cluster membership but not relative position of clusters 
    \item the observation that false clustering may occur - one should thus run t-SNE multiple times if possible
\end{itemize}
More recently, the focus of theoretical research on t-SNE has shifted to the question of equilibrium distributions and convergence in the large data limit ($n \to \infty$). 
\cite{murray2024largedatalimitsscaling} shows that standard t-SNE embeddings do not have a consistent limit as $n \to \infty$ and proposes a rescaled model with a consistent limit which mitigates the asymptotic decay of the attractive forces. 

\section{Clustering Guarantees}
In this section, we will present the result in \cite{LinStei22} and \textcolor{red}{TODO: maybe investigate whether the assumptions hold for real datasets}. 

The following assumptions are made: 
\begin{enumerate}
    \item $\mathcal{X}$ is clusted. We assume that there exists a number of clusters $k \in \mathbb{N}$ and a map $\pi: \{1,\dots,n\} \to \{1,\dots,k\}$ which maps each point to a cluster, such that the following holds: if $\pi(x_i) = \pi(x_j)$, then \begin{equation}
        p_{ij} \geq \frac{1}{10 n |\Omega(i)|}
    \end{equation}
    where $\Omega(i)$ is the size of the cluster in which $x_i$ and $x_j$ lie. 
    \item Parameter choice. Step-size $h$ and exaggeration parameter $\alpha$ are chosen such that for some $1\leq i \leq n$, 
    \begin{equation}
        \frac{1}{100} \leq \alpha h \sum_{j \neq i, \pi(i) = \pi(j)} p_{ij} \leq \frac{9}{10}
    \end{equation}
    \item Localized initialization. The initialization of the low-dimensional embedding satisfied $\mathcal{Y} \in [-0.01, 0.01]^2$. 
\end{enumerate}
The main result is applicable to a single cluster. 

\begin{thm}[\cite{LinStei22}]
    Under assumptions (i)-(iii), the diameter of the embedded cluster $\{y_j: 1 \leq j \leq n \land \pi(i) = \pi(j)\}$ decays exponentially (at universal rate) until its diameter satisfies, for some constant $c > 0$,
    \begin{equation}
        \text{diam} \{y_j: 1 \leq j \leq n \land \pi(i) = \pi(j)\} \leq c \cdot h \left(\alpha \sum_{j \neq i \pi(i) = \pi(j)} p_{ij} + \frac{1}{n} \right). 
    \end{equation}
\end{thm}
\textcolor{red}{Should I include a proof? Or at least its idea?}

There are several problems with this result however. 
\begin{itemize}
    \item the assumptions on $h$ and $\alpha$ do not really make sense with what has been established empirically. In this paper, they suggest $h=1$, whereas later research has suggested values like $h=200$ or higher for large datasets. 
    \item There is this paper which says that condition (i) does not hold for datasets in practice even if they are clusterable. Steinerman says it's only a loose condition, but this doesn't seem to actually be the case. 
\end{itemize}

\section{Large Data Limits}
For this section, instead of assuming a given dataset, we have a probability distribution $\mu \in \mathcal{P}(\Omega)$ on $\mathbb{R}^d$ with support on $\Omega$, a bounded and $C^2$ domain. 
We further assume that $\mu$ has a bounded density function $\rho(x)$ with respect to the Lebesgue measure. 
Our points $X_1, \dots, X_n$ are then drawn independently from $\mu$ and the $p_{ij}$ and $q_{ij}$ values are calculated as always. 
If $n$ datapoints are drawn, we define the matrices $P_n = (p_{ij})_{i,j=1}^n$ and $Q_n = (q_{ij})_{i,j=1}^n$. 

Since we want to study large data limits, we do not consider the KL-divergence between $P_n$ and $Q_n$ directly, but instead define a closely related functional 
\begin{equation}
    \text{KL}_n (T) = \sum_{i,j} p_{ij} \log \frac{q_{ij}}{q_{ij}(T)}
\end{equation}
where 
\begin{equation}
    q_{ij}(T) \coloneq \frac{(1+ |T(X_i) - T(X_j)|^2)^{-1}}{\sum_{k \neq l} (1+ |T(X_k) - T(X_l)|^2)^{-1}}
\end{equation}
and $T: \mathbb{R}^d \to \mathbb{R}^m$. This is now technically not a KL-divergence anymore. 

Building on the view of t-SNE in terms of attraction-repulsion dynamics, \cite{murray2024largedatalimitsscaling} splits up the t-SNE objective function into an attractive term $A_n[T]$, a repulsive term $R_n[T]$ and a purely data-dependent term  $D_n$ (which plays no role during gradient descent). Defining 
\begin{equation}
    A_n[T] \coloneq \frac{1}{n} \sum_{i=1}^n \frac{\sum_{j=1}^n \exp(-|X_i - X_j|^2/(2\sigma_i^2)) \log(1+ |T(X_i) - T(X_j)|^2)}{\sum_{j=1}^n \exp(-|X_i - X_j)^2 / (2\sigma_i^2)}
\end{equation}
and 
\begin{equation}
    R_n[T] \coloneq \log \left( \frac{1}{n^2} \sum_{i=1}^n \sum_{j=1}^n \frac{1}{1+ |T(X_i) - T(X_j)|^2} \right), 
\end{equation}
we can write $\text{KL}_n (T) = A_n[T] + R_n[T] + D_n$. 

This means, we have the following optimization problem
\begin{equation}
    {\arg \min}_{T: \mathbb{R}^d \to \mathbb{R}^m} \text{KL}_n (T) = {\arg \min}_{T: \mathbb{R}^d \to \mathbb{R}^m} A_n[T] + R_n[T].
\end{equation}

Again, we observe that there is a competition between minimizing the attractive term and minimizing the repulsive term. 

On Perplexity: They allow perplexity to grow slowly with the number of samples $n$. 

We now define an averaged version of the energies, where we replace the stochastic $\sigma_i$ with the deterministic $h_n \sigma_\kappa(x)$: 
\begin{equation}
    \tilde{A}_n[T] \coloneq  \int_{\Omega} \frac{\int_{\Omega} \exp(-|x - x'|^2/(2 h^2 \sigma_\kappa^2(x))) \log(1+ |T(x) - T(x')|^2) \rho(x')dx'}{\int_{\Omega} \exp(-|x - x'|^2/(2 h^2 \sigma_\kappa^2(x))) \rho(x')dx'} \rho(x)dx  
\end{equation}
and 
\begin{equation}
    \tilde{R}_n[T] \coloneq \log \left( \int \int_{\Omega \times \Omega} \frac{1}{1+ |T(x) - T(x')|^2} \rho(x) \rho(x')dx dx' \right), 
\end{equation}

\textcolor{red}{TODO: make the equations look nicer}

This first result states that the averaged t-SNE energy does not have a limiting solution. 
\begin{thm}[\cite{murray2024largedatalimitsscaling}]
    Let $T_n$ be a sequence of minimizers of the energies $\tilde{A}_{h_n}[T] + \tilde{R}[T]$. Then $T_n$ does not converge pointwise to any $T^* \in L^\infty(\Omega, \mathbb{R}^m)$. 
\end{thm}


\cleardoublepage        % Kapitel immer rechts beginnen
\chapter{Practical Aspects}\label{chapter:practice}

\section{Accelerating t-SNE}

The original t-SNE algorithm is not very fast. Its runtime is $\mathcal{O}(N^2)$, which quickly becomes infeasible for datasets with $100,000$ or more points \textcolor{red}{check the exact number} \comment{explain more why this runtime}. In this section, we will give an overview of the two most popular methods proposed to speed up the algorithm. 

\subsection*{Barnes-Hut t-SNE}
This method uses tree-bases algorithms to speed up t-SNE and was proposed by van der Maaten in 2014 \cite{vdMaa14}. 

\begin{itemize}
    \item the gradient of the t-SNE loss function (KLD) has a natural interpretation as an $N$-body system in which all of the $N$ points in the low-dimensional embedding exert forces on each other. We then need to compute the resultant force on each of the points
\end{itemize}

%\title{Approximating Attractive Forces}
Computing the attractive force is not too expensive computationally, if we approximate input similarities and use vantage-point trees. Recall that input similarities $p_{ij}$ are computed based on a Gaussian kernel. 
Thus, $p_{ij}$ values corresponding to dissimilar input objects $x_i$ and $x_j$ are very small. 
So, it makes sense to develop a sparse approximation for the $p_{ij}$. Instead of computing $N^2$ pairwise similarities, we focus on the $\lfloor 3 \kappa \rfloor$ nearest neighbors of each of the $n$ input objects only, where $\kappa$ denotes the perplexity. We denote the nearest neighbor set of $x_i$ by $\mathcal{N}_i$. 

The similarities are thus given by 
\begin{equation}
    p_{j|i} = \begin{cases}
    \frac{\exp(-\norm{x_i - x_j}^2) / 2 \sigma_i^2}{\sum_{k \in \mathcal{N}_i} \exp(-\norm{x_i - x_k}^2) / 2 \sigma_i^2} & \text{ if } j \in \mathcal{N}_i \\
    0  & \text{ otherwise}
    \end{cases}
\end{equation}
which are again symmetrized then. 

We can find the nearest neighbor sets $\mathcal{N}_i$ in $\mathcal{O}(u N \log N)$ time by building a data structure called a vantage-point tree and performing nearest neighbor search with its help. For details, see \cite{vdMaa14}. 

\textcolor{red}{TODO: maybe say more about vantage-point trees here?}

%\title{Approximating Repulsive Forces}
Naively computing the repulsive forces is not a good idea. It would be in $\mathcal{O}(N^2)$. 
Using the Barnes-Hut algorithm however, this can be sped up to $\mathcal{O}(N \log N)$. 
The algorithm relies on the observation that the repulsive forces exerted between two small groups of points are very similar whenever these two groups are relatively far away from each other. 
More precisely, if we consider points $y_i, y_j$ and $y_k$ with $\norm{y_i - y_j} \approx \norm{y_i - y_k} \gg \norm{y_j - y_k}$, then the contributions of $y_j$ and $y_k$ to $F_{\text{rep}, i}$ will be roughly equal. 

To take advantage of this fact, the Barnes-Hut algorithm constructs a data structure known as a quadtree (or for three-dimensional embeddings an octtree). 
Once we have built a quadtree on the current embedding, we traverse it via depth-first search and decide at every node of the quadtree, whether the corresponding cell can be used as a summary for the contributions to $F_{\text{rep}}$ of all points in the cell or if we need to go deeper. 

\textcolor{red}{Maybe insert a picture of a Quadtree here? But I am not sure how much to write about Barnes-Hut really...}

\subsection*{Fast Interpolation-Based t-SNE}
As observed in \cite{KoBe19SingleCell}, the Barnes-Hut implementation of t-SNE, while faster than the originally proposed algorithm, still becomes provibitively slow for datasets with $n \gg 100,000$. 
The FFT accelerated, fast interpolation-based version of t-SNE proposed by Linderman et. al. in 2019 \cite{Lin19} attempts to further speed up the computation of t-SNE embeddings with a special view towards applications in the analysis of single-cell RNA-seq data, which keeps growing and datasets often have north of 1 million datapoints. 

Rough notes from \cite{Lin19}: 
\begin{itemize}
    \item the repulsive force acting on each point can be expressed via sums of the form \begin{equation}
        \phi(y_i) = \sum_{j=1}^N K(y_i, y_j) q_j 
    \end{equation}
    with either $K_1(y,z) = \frac{1}{1+\norm{y-z}^2}$ or $K_2(y,z) = \frac{1}{(1+\norm{y-z}^2)^2}$ (both of these are smooth and translation-invariant kernels). 
    \item key idea: instead of computing all these sums (this would be in $\mathcal{O}(N^2)$), we instead use polynomial interpolation to approximate the kernels $K$ 
    \item interpolation strategy: subdivide the embedding space into a number of intervals (in the 1D case) or squares (in the 2D case)
    \item in each interval, choose a small, fixed number $p$ of equispaced interpolation nodes 
    \item approximate the kernel $K$ by its low-order polynomial interpolant $K_p$ over these nodes 
    \item This replaces the direct kernel evaluation between every pair with an evaluation on a much coarser grid.
\end{itemize}
TLDR: Low-order polynomial interpolation mediates interactions through a fixed number of nodes, FFT exploits Toeplitz structure to accelerate convolution sums, so we get a near linear runtime. 


% algorithm pseudocode here
\begin{algorithm}[H]
    \caption{FFT-accelerated Interpolation-based t-SNE (FIt-SNE)}
    \label{alg:fit-sne}
    \KwIn{embedding points $\{y_i\}_{i=1}^N$, interpolation coefficients $\{q_i\}_{i=1}^N$, number of intervals $N_{\text{int}}$, interpolation points per interval $p$}
    \KwOut{$\varphi(y_i) = \sum_{i=1}^N K(y_i, y_j) q_j$ for $i=1, \dots, N$}


    For every interval $I_l$, form $p$ equispaced nodes $\tilde{y}_{j, l} = 1/(2N_{\text{int}}p) + \frac{j-1 + (l-1)p}{N_{\text{int}}p}$ for $j=1,\dots,p$. 

    \For{$I = 1$ \KwTo $N_{\text{int}}$}{
        Compute the coefficients $w_{m,l}$ given by 

        \[
            w_{m,l} = \sum_{y_i \in I_l} L_{m, \tilde{y}^l}(y_i) q_i, \text{      } m=1,\dots,p
        \]
    }

    Use FFT to compute values of $v_{m,n}$ given by 

        \[
            \begin{pmatrix}
                v_{1,1} & v_{2,1} & \dots & v_{p, N_{\text{int}}}
            \end{pmatrix}^T = \tilde{K} \begin{pmatrix}
                w_{1,1} & w_{2,1} & \dots & w_{p, N_{\text{int}}}
            \end{pmatrix}^T
        \]
    where $\tilde{K}$ is the Toeplitz matrix given by $\tilde{K}_{ij} = K(\tilde{y_i}, \tilde{y_j})$, $i, j = 1, \dots, N_{\text{int}}p$.
    
    \For{$I = 1$ \KwTo $N_{\text{int}}$}{
        Compute $\varphi(y_i)$ at all points $y_i \in I_l$ via 

        \[
            \varphi(y_i) = \sum_{j=1}^p L_{j, \tilde{y}^l}(y_i) v_{j,l}
        \]
    }

\end{algorithm}

\subsection*{Comparing Barnes-Hut and FIt-SNE}
When comparing the two methods, we only see minor differences in the embedding. 
However, the FIt-SNE version is many times faster than Barnes-Hut t-SNE, which is why we will be using FIt-SNE in all of our experiments going forward. 

As pointed out in \cite{openTSNE}, FIt-SNE scales linearly with the number of smaples, but it introduces additional computational overhead for each embedding.
This is often unneccessary for smaller datasets, which is why openTSNE uses Barnes-Hut t-SNE for data sets with fewer than $10,000$ samples and FIt-SNE for data sets with at least $10,000$ datapoints. 
%\begin{figure}[h]
  %  \centering 
  %  \includegraphics[width=\linewidth]{figures/BH_vs_FFT.png}
 %   \caption{Barnes-Hut versus FIt-SNE on the flow18 dataset}
 %   \label{fig:BH_vs_FIt-SNE}
%\end{figure}


\section{Optimizing t-SNE Parameters and Hyperparameters}

\subsection{Perplexity}
Let us come back to the variance $\sigma_i$ of the Gaussian centered at each datapoint $x_i$. What is a good value to choose? 
If we fixed a single value $\sigma$ to be the same for every datapoint, this is likely not a good choice, because real-life data often does not have a constant density everywhere but instead has sparser and denser regions. 
Given that we want to consider approximately the same number of nearest neighbors for each $x_i$, we opt to choose larger values of $\sigma_i$ for sparse regions and smaller bandwidths for dense regions. 

\textcolor{red}{\textbf{TO DO}: find a better way to describe what perplexity actually does, the following is just taken directly from \cite{vdMaa08}}

Any particular value of $\sigma_i$ induces a probability distribution $P_i$ over all other datapoints. 
The entropy of this distribution increases as $\sigma_i$ increases. 
The user can specify a specific so-called perplexity
\begin{equation}
    \kappa = \text{Perp}(P_i) = 2^{H(P_i)} 
\end{equation}
where $H(P_i) = -\sum_{j} p_{ij} \log_2 p_{ij}$ denotes the Shannon entropy of $P_i$. 

\textcolor{red}{\textbf{TO DO}: do we use $p_{ij}$ or $p_{j|i}$ in the definition of perplexity?}

Then, t-SNE performs a binary search for the value of $\sigma_i$ that produces the user-specified perplexity. 

One can think of perplexity as a smooth measure of the effective \textcolor{red}{\textbf{TO DO}: what does this actually mean?} number of neighbors being considered in the calculation of the $p_{ij}$. As such, larger perplexity values are computationally more expensive. 

An important question is: which values are good perplexity values. Several suggestions have been made \textcolor{red}{\textbf{TO DO}: compile research on suggestions here}. 


\subsection{Initialization}
The standard t-SNE algorithm starts with an initization $y_i^{(0)}$ for $i=\{1, 2, \dots, n\}$ which are drawn independently from $\mathcal{N}(0, \delta^2 I)$ for some small $\delta > 0$, see \cite{vdMaa08} and \cite{vdMaa14}. 

However, recent work \cite{kobak21} has shown that informative initialization leads to embeddings that better preserve large-scale structures within the data. 
They argue that using informative intialization should be the default option for t-SN
E.
Indeed, modern implementations of t-SNE in libraries like openTSNE \cite{openTSNE} or Scikit-learn now all use PCA initialization by default. 
This means that we perform a principal component analysis on the input data $x_1, \dots, x_n$ and use the output $y_1, \dots, y_n$ as the initial points for the low-dimensional embedding. 

\begin{figure}[h]
    \centering 
    \includegraphics[width=\linewidth]{figures/pca_vs_random_init.png}
    \caption{PCA versus random initialization}
    \label{fig:PCA_vs_random}
\end{figure}
In our experiments, we do not necessarily see a visual improvement of the PCA initialized embedding over the random initialization one, but \textcolor{red}{one should look at other metrics to see how well the global structure is preserved e.g. Pearson correlation or Spearman correlation}. 

\subsection{Early Exaggeration}
Early exaggeration was first proposed as a method of optimizing t-SNE in \cite{vdMaa08}. They proposed multiplying all the $p_{ij}$ by a value $\alpha > 0$ for the first few iterations of the algorithm. Since our loss function encourages the $q_{ij}$ to model the $p_{ij}$ as closely as possible, we achieve artificially large $q_{ij}$ values this way. This means that relatively tight clusters are being formed, which can then move around more easily in space, making it easier to find a good global organization of the clusters. 

Open question: What should $\alpha$ be and for how many iterations should we keep EE on? 
\begin{itemize}
    \item \cite{vdMaa08} originally proposed $\alpha = 4$, for $50$ iterations out of $1000$ in total 
\end{itemize}

We can also understand early exaggeration from a dynamical systems viewpoint. 
We start by recalling the gradient of the t-SNE cost function: 
\begin{equation}
    \frac{\partial C}{\partial y_i} = 4 \sum_{j \neq i} (p_{ij} - q_{ij}) q_{ij} Z (y_i - y_j)
\end{equation}
with normalization term $Z = \sum_{k \neq l} (1+ \norm{y_k - y_l}^2 )^{-1}$. Notice that we can split the gradient into two parts 
\begin{equation}
    \frac{\partial C}{\partial y_i} = 4 (F_{\text{attr}} + F_{\text{rep}}) = 4 \left( \sum_{j \neq i } p_{ij} q_{ij} Z (y_i - y_j) - \sum_{j \neq i} q_{ij}^2 Z (y_i - y_j) \right) 
\end{equation}
where $F_{\text{attr}}$ denotes the sum of all attractive forces and $F_{\text{rep}}$ the sum of all repulsive forces. 

This next explanation is taken from \cite{LinStei22}.

Why does it make sense to call these attractive and repulsive forces? Since we want to minimize the cost function, we perform gradient descent and step in the direction of the negative gradient, so we consider the term
\begin{equation}
   - \frac{1}{4} \frac{\partial C}{\partial y_i} = \sum_{j \neq i } p_{ij} q_{ij} Z (y_j - y_i) - \sum_{j \neq i} q_{ij}^2 Z (y_j - y_i).  
\end{equation}
The first term is considered the attractive term, since it moves the point $y_i$ towards a weighted average of the other $y_i$. 
The weights $p_{ij} q_{ij} Z$ are bigger if the two points are close to each other (both in the low- and high-dimensional space). 
The second term has the opposite sign and thus pushes $y_i$ away from a weighted average of the other points. This time, however, the weights only depend on the closeness of points in the low-dimensional space. 
Put together, this means that the attractive term attracts points that are actually meant to be with each other (based on their similarity in the high-dimensional space) and the repulsive term pushes points apart that get too close in the embedding space, regardless of their real similarity. 



\cleardoublepage        % Kapitel immer rechts beginnen
\chapter{Experiments}\label{chapter:experiments}
in which we try to replicate results from \cite{belkina19}. 

\section{Effect of Varying Perplexity Values}
\begin{figure}[ht]
    \centering 
    \includegraphics[width=\linewidth]{figures/perp_flow_001.png}
    \caption{Different perplexity values on 0.1 percent the flow18 dataset}
    \label{fig:perp001}
\end{figure}

\begin{figure}[ht]
    \centering 
    \includegraphics[width=\linewidth]{figures/perp_flow_01.png}
    \caption{Different perplexity values on 1 percent the flow18 dataset}
    \label{fig:perp01}
\end{figure}

\begin{figure}[ht]
    \centering 
    \includegraphics[width=\linewidth]{figures/perp_flow_1.png}
    \caption{Different perplexity values on 10 percent the flow18 dataset}
    \label{fig:perp1}
\end{figure}

\begin{figure}[ht]
    \centering 
    \includegraphics[width=\linewidth]{figures/perp_flow_5.png}
    \caption{Different perplexity values on 50 percent the flow18 dataset}
    \label{fig:perp5}
\end{figure}

\cleardoublepage        % Kapitel immer rechts beginnen
\chapter{Discussion}\label{chapter:discussion}

As pointed out in \cite{kobak21}, \enquote{it remains challenging to measure how faithful a given embedding is}.  


Mention the difficulty of interpreting t-SNE plots \cite{Wa16Distill}. 

%\cleardoublepage
%\input{chapter_appendix/chapter_appendix}

% Weitere Symbole für das Symbolverzeichnis
\symbolindex[f]{$F$}{A topological or Riemann surface.}{}

% Symbolverzeichnis
\cleardoublepage        % Auch diese sollen auf der rechten Seite beginnen
\printnomenclature      % Symbolverzeichnis ausgeben

% Stichwortverzeichnis
\cleardoublepage        % Auch diese sollen auf der rechten Seite beginnen
\printindex             % Stichwortverzeichnis ausgeben

% Referenzen
\nocite{*}              % Alle Einträge der Bib-Datei sollen in die Referenzen
\cleardoublepage        % Auch diese sollen auf der rechten Seite beginnen
\printbibliography      % Bibliographie ausgeben.

\end{document}