\chapter{Dimensionality Reduction}\label{chapter:dimensionality-reduction}
Notes from \cite{vdMaa08}: 
\begin{itemize}
    \item aim of dimensionality reduction: preserve as much of the significant structure of the high-dimensional data as possible in the low-dimensional map 
    \item there have been various proposals that differ in the type of structure they preserve 
    \item linear techniques: PCA and classical MDS focus on keeping low-dimensional representations of dissimilar datapoints far apart - PCA maximises variance 
    \item often though, high-dimensional data does not lie on a linear space but instead on or near a non-linear (low-dimensional) manifold. here, it is usually more important to keep the low-dimensional representations of very similar datapoints close together. this is something that cannot be done with a linar mapping. 
\end{itemize}


- talk a bit about PCA here 
- linear and nonlinear dimensionality reduction 
- why do we need it? 