\chapter{Introduction}
Notes from \cite{vdMaa08}: 
\begin{itemize}
    \item t-SNE is capable of capturing both local and global structure reasonably well 
\end{itemize}
A sentence I like from \cite{vdMaa14}: \enquote{Visual exploration is an essential component of data analysis, as it allows for the development of intuitions and hypotheses for the processes that generated the data.}


Maybe it would be good to say something about the importance of visualization data in general here. 

Notes from \cite{LinStei22}:
\begin{itemize}
    \item traditional approaches to data analysis and visualization often fail in the high-dimen\-sio\-nal setting or it is impossible to visualize more than two or three dimensions (think boxplots, traditional scatterplots etc.)
    \item goal of t-SNE: simplify identification of clusters 
    \item t-SNE is used in the standard analysis pipeline for single-cell RNA sequencing data nowadays
\end{itemize}

Notes from \cite{vdMaa14}:
\begin{itemize}
    \item In high-dimensional spaces, only small pairwise distances are reliable. This is why most visualization techniques for high-dimensional data only try to model these accurately. 
    \item a high-level view of t-SNE: the algorithm minimizes a divergence betweeen a distribution that measures pairwise similarities of input objects and a distribution that measures pairwise similarities of the corresponding low-dimensional points in the embedding 
\end{itemize}

Not sure if this fits into the introduction, but I quite like this explanation of t-SNE by \cite{KoBe19SingleCell}: t-SNE essentially lets points interact like physical particles governed by the two following laws: all points repell each other and each point is attracted to its nearest neighbor.

Another interesting note from \cite{KoBe19SingleCell}: t-SNE is often perceived as having only one free parameter to tune, namely perplexity. But under the hood there are so many other optimization parameters (learning rate, number of iterations, EE factor etc.) which can have a large impact on the quality of the embedding! 