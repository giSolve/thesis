\chapter{Multilevel quadrature of eigenspaces}\label{chapter:multilevel-quadrature}

%\comment{Do I want to restrict myself to this or do I want to take nonlinear eigenvalue problems that depend on a random paramenter in general?}
Now, we recall the original problem (\ref{eq:og-prob}).
We assume that the domain $D(\bs{y})$ is given as a random deformation of a reference Lipschitz domain $D_{\text{ref}}\subset \bR^3$ with piecewise smooth surface $S_{\text{ref}} \coloneq \partial D_{\text{ref}}$.
Further, we assume that this deformation can be encoded in a parameter $\bs{y} \in [-1,1]^{\bN}$ (cf. \cite{DOLZ2022114242}).
We will approximate the parameter space $[-1,1]^{\bN}$ by a finite dimensional space $[-1,1]^N$ where $N$ is large enough.
Our aim is to compute stochastic quantities of interest of the eigenpairs, namely their expectation.
As our parameter space is high dimensional, we want to use a quasi-Monte Carlo method, which is preferred to Monte Carlo due to the higher convergence rate.

When employing sampling-based numerical methods to calculate the expectation, such as quasi-Monte Carlo, the challenge of relating eigenvalues to each other across the samples arises.
Even if the trajectories of the eigenvalues are given by continuously differentiable paths, it is not evident how to track these over the discrete samples, especially when the random parameter is high dimensional, as crossings and bifurcations could occur.
We use an approach that exploits the more meaningful stochastic properties of the eigenspaces \cite{Doelz_Ebert_2024}.

Our goal is to compute the expectations of the eigenvalues that lie within a given contour $\Gamma \subset \bC$.
In order for the method to work, we require that the eigenvalues that lie inside of $\Gamma$ are separated by our contour over the entire parameter space from the rest of the eigenvalues (cf. \cite{kato_perturbation,grubivsic2023stochastic}).

%\section{Modelling of random domains and surfaces}

%%CHANGE OMEGA 
In what follows, let $D_{\text{ref}} \subset \bR^3$ denote a Lipschitz domain with piecewise smooth surface $S_{\text{ref}} \coloneq \partial D_{\text{ref}}$ and $(\Omega, \mathcal{F}, \bP)$ be a complete probability space.
We assume the uncertainty in the obstacle to be encoded by a random domain field \comment{meter cita}.
Hence, we assume the existence of a uniform $\mathcal{C}^1$-diffeomorphism $\mathbf{\chi}_D \colon \closure{D}_{\text{ref}} \times \Omega \to \bR^3$, i.e.
\[
    \lVert \mathbf{\chi}_D (\omega)\rVert _{\mathcal{C}^1(\closure{D}_{\text{ref}};\bR^3)}, \lVert \mathbf{\chi}_D\inv (\omega)\rVert _{\mathcal{C}^1(\closure{D(\omega)};\bR^3)} \leq C  \quad \text{for } \bP\text{-a.e. } \omega \in \Omega
\]
such that
\[
    D(\omega) = \mathbf{\chi}_D (D_{\text{ref}}, \omega).
\]

meter más de lo de cómo pasar al parameter $y\in [-1,1]^N$.
\section{Single-level Monte Carlo quadrature}
We introduce our method first for the single-level case.
Consider nonlinear eigenvalue problems as in \Cref{chapter:cim} now for a parametrized family of functions $T_{\bs{y}} \in H(\Omega, \bC^{m \times m})$ for $\bs{y}\in [-1,1]^{\bN}$, where we take $T_\mathbf{0}$ as the reference element.
Let $k$ be the number of eigenvalues, including multiplicities, inside of the contour $\Gamma$.
We write $A_{\bs{y}, 0}$ and $A_{\bs{y}, 1}$ if \cref{eq:A0,eq:A1} are computed for $T_{\bs{y}}$, respectively.

%For each sample $\bfy \in [-1,1]$, we want to find a suitable quantity to compute for which a meaningful expectation can be found.
We recall \Cref{section:reduction-cim}, where the method to solve nonlinear eigenvalue problems was presented.
After calculating the matrices $A_{\bs{y}, 0}$ and $A_{\bs{y}, 1}$, we computed the eigenvalue decomposition 
\begin{equation}
    \label{eq:svd-A0-param}
    A_{\bs{y}, 0} = V_{\bs{y},0} \Sigma_{\bs{y},0} W_{\bs{y}, 0}\hm.
\end{equation}
%\comment{pensar si quiero resumir aquí brevemente la herleitung, bzw. poner tipo la fórmula de B y de S}.
Now, the central idea of the method is to fix two orthogonal matrices $V_0 \in \bC^{m\times k}$ and $W_0 \in \bC^{l\times k}$, which will replace $V_{\bs{y},0}$ and $W_{\bs{y}, 0}$.
Instead of the singular value decomposition \eqref{eq:svd-A0-param}, we then take
\begin{equation}
    A_{\bs{y}, 0} = V_0 \Sigma_{\bs{y},0} W_0\hm,
\end{equation}
where $\Sigma_{\bs{y},0} = V_0\hm A_{\bs{y}, 0} W_0$ is not necessarily a diagonal matrix anymore.
We see that we can still perform each step of the reduction with this modified decomposition, as the only property needed of $V_{\bs{y},0}$ was that it be orthogonal, so we get
\[ 
    B_{\bs{y}} = V_0\hm A_{\bs{y},1} W_0 \Sigma_{\bs{y},0}\inv = S_{\bs{y}}\Lambda_{\bs{y}} S_{\bs{y}}\inv = V_0\hm V_{\bs{y}} \Lambda_{\bs{y}} (V_0\hm V_{\bs{y}})\inv.
\]
%\comment{y poner aquí las fórmulas de B y S otra vez pero con el $V_0$ fijado}
So, essentially, the major modification is that the matrix of eigenvectors of $B$ is given by $S_{\bs{y}} = V_0\hm V_{\bs{y}}$ instead of $S_{\bs{y}} = V_{\bs{y},0}\hm V_{\bs{y}}$, which means that we are always relating the matrix of eigenvectors of $T_{\bs{y}}$ that is given by $V_{\bs{y}}$ to the same orthogonal matrix $V_0$ instead of taking a different one for each sample.
%\comment{revisar/mejorar esto de qué es la idea}
Then, we compute the expectation of the matrix $B$, which has the same eigenvalues including multiplicities as our original problem, and from which we can directly derive eigenvectors for our original problem, see \Cref{thm:equiv-B-T(z)}.
We summarize the method in \Cref{alg:single-level}.
%\comment{add better explanation of why we do this, why this is a suitable größe}

Let $l \geq k$.
\begin{algorithm}
    \DontPrintSemicolon

    \SetKwInput{Input}{Input}
    \SetKwInput{Output}{Output}
    \SetKw{KwGoTo}{go to}

    \Input{A family $T_{\bs{y}} \in H(\Omega, \bC^{m \times m})$ of parametrized matrices and a contour $\Gamma \subset \Omega$}
    %\Output{An approximation of $\bE[B]$, where $B$... (retains eigenvalues and mulitplicity o así)}
    \Output{An approximation of $\bE[B]$}
    \BlankLine
    Draw a random matrix $\widehat{V} \in \bC^{m\times l}$\;
    Choose two orthogonal matrices $V_0$ and $W_0$ (e.g. using the SVD of $A_{\mathbf{0},0})$\;
    \For{$i = 1,\ldots,M$}{
        Draw a random sample $\bs{y}_i \sim \text{Unif}([-1,1]^N)$\;\label{line:monte-carlo-y}
        Compute $A_{\bs{y}_i, 0}$ and $A_{\bs{y}_i, 1}$\;
        Set $\Sigma_{\bs{y}_i,0} = V_0 A_{\bs{y}_i,0} W_0\hm$\;
        Set $B_{\bs{y}_i} = V_0\hm A_{\bs{y}_i,1} W_0 \Sigma_{\bs{y}_i,0}\inv$\;
    }
    \Return{$B = \frac{1}{M} \sum_{i=1}^M B_{\bs{y}_i}$}
    \caption{Single-level Monte Carlo quadrature for nonlinear eigenvalue problems}\label{alg:single-level}
\end{algorithm}
\begin{rem}
    (a) As $V_0$ and $W_0$ are not the matrices associated to the eigenvalue decomposition of $A_{\bfy_i, 0}$, the matrix $\Sigma_{\bfy_i, 0}$ is not necessarily diagonal anymore.
    Therefore, one should not invert the matrix directly.
    Instead, it is possible to compute $(W_0 \Sigma_{\bfy_i,0}\inv)\tp$ as the solution $X\tp$ of the system of linear equations $\Sigma_{\bfy_i,0}\tp X\tp = W_0\tp$.

    (b) In the current form, the algorithm performs a Monte Carlo quadrature (see \Cref{line:monte-carlo-y}).
    It is also possible to replace it by a quasi-Monte Carlo method. %\comment{igual meto directamente la quasi Monte Carlo methode en el algoritmo, pero entonces sería un poco más unübersichtlich}
\end{rem}
\section{Multilevel Monte Carlo quadrature}
For a more efficient calculation of the quantities of interest, it is also possible to use a multilevel quadrature method \cite{DOLZ2022114242,Giles_2015}.
The main computational effort when considering a stochastic Dirichlet Laplacian eigenvalue problem lies in computing the approximation $\underline{V}(\kappa)$ of the single layer boundary integral operator.
The idea of multilevel Monte Carlo is to perform most simulations with low accuracy at a corresponding low cost and rather few simulations at high accuracy and high cost.
In our case, low accuracy corresponds with using a Galerkin approximation space on a coarser refinement level and accordingly, high accuracy is achieved when using an approximation space with a higher refinement.

Using a multilevel quadrature approach, we obtain
\begin{equation}
    \label{eq:B-multilevel}
    \bE[B] \approx \mathcal{Q}_L^{\text{ML}} [B] = \sum_{\ell=0}^L \mathcal{Q}_{L-\ell} [B^{(\ell)}-B^{(\ell-1)}],
\end{equation}
where $\mathcal{Q}_{\ell}$ is a quadrature rule on level $\ell$.
The approximation $B^{(\ell)}$ is computed using the Galerkin approximation $\underline{V}^{(\ell)}(\kappa)$ of the single layer boundary integral operator of the Helmholtz equation on refinement level $\ell$, setting $B^{(-1)} = 0$. For the approximation error of the multilevel quadrature, it holds a sparse tensor product-like error estimate.
\iffalse
If $\varepsilon_{\ell} \to 0$ is a monotonically decreasing sequence with $\varepsilon_{\ell} \cdot \varepsilon_{L-\ell} = \varepsilon_L$ for every $L\in \bN$ and
\[
    \lVert \mathcal{B}_{L-\ell}-\bE[B]\rVert \leq c_1\varepsilon_{L-\ell} \quad \text{and}\quad \lVert B^{(\ell)}-B\rVert \leq c_2\varepsilon_{\ell}
\]
for some suitable norms and constants $c_1,c_2 > 0$, then
\[
    \lVert \mathcal{Q}_L^{\text{ML}}[B] - \bE[B] \rVert \leq C\varepsilon_L
\]
for a constant $C>0$, \comment{what conditions do we need?}. %given that $B$ is sufficiently regular \comment{revisar esto, tipo cuáles son las condiciones}.
\fi

%\comment{igual conviene otra letra que ml es también la mulitplicity}
Since the size of $\underline{V}^{(\ell)}(\kappa) \in \bC^{m_{\ell} \times m_{\ell}}$ depends on the refinement level, we cannot choose the orthogonal matrices $V_0$ and $W_0$ independent of the level, but need matrices $W_0^{(\ell)}$ and $V_0^{(\ell)}$.
%In particular, this means that the eigenspaces that we fix are not the same across all levels.
If, on a single level, we choose $V_0$ and $W_0$ using the SVD of $A_{\mathbf{0},0} = V_0 \Sigma_0 W_0\hm$, $W_0$ and $V_0$ depend only on $\underline{V}^{(\ell)}(\kappa)$ and $\widehat{V}$.
As we cannot control $\underline{V}(\kappa)$, we consider $\widehat{V}$ instead.

For the contour integral method, the only property required of $\widehat{V}$ is that it has full rank.
Furthermore, for the computation of $A_{\mathbf{0},0}$ and $A_{\mathbf{0},1}$, we are computing $\underline{V}(\kappa)\inv \widehat{V}$.
Since $\underline{V}(\kappa)$ is the system matrix of a Galerkin method, we can interpret $\widehat{V}$ as a matrix of discretized linear forms.
So, instead of choosing the columns of $\widehat{V}$ randomly, we can choose the linear forms randomly in a suitable sense such that $\widehat{V}$ has full rank independently of the ansatz space.
Once we obtain $\widehat{V}^{(\ell)}$ in this fashion, we can compute $V_0^{(\ell)}$ and $W_0^{(\ell)}$ on each level as before.

Regarding the choice of the linear forms, we propose the following.
For each column of $\widehat{V}$, we take a suitable random function that we test against the corresponding ansatz space with the $L^2$-scalar product to get discretized linear forms for each level.
A way of finding appropiate random functions is to choose functions $f(\bfx) = e^{-\frac{\norm{\bfx - \bfm}_2}{\sigma}}$, where $\bfm \in \bR^3, \sigma >0$ are random parameters.

\Cref{alg:multilevel} shows an example of how a multilevel approach for our problem could be implemented.
%\comment{ask about inputs/outputs, what exactly to say there, do I want to take $T$ there or $\underline{V_{\kappa}}$}

Let $l' \geq k$.
\begin{algorithm}
    \DontPrintSemicolon

    \SetKwInput{Input}{Input}
    \SetKwInput{Output}{Output}
    \SetKw{KwGoTo}{go to}

    \Input{Aproximations $T_{\bs{y}}^{(\ell)} \in H(\Omega, \bC^{m \times m})$ of a family of parametrized functions for $\ell = 0,\ldots,L$ and a contour $\Gamma \subset \Omega$}
    \Output{A multilevel approximation $\mathcal{Q}_L^{\text{ML}}[B]\approx \bE[B]$}%, where $B$... (retains eigenvalues and multiplicity o así)}
    \BlankLine
    Generate $l'$ random functions $f_j(\bfx) = e^{-\frac{\norm{\bfx - \bfm_j}_2}{\sigma_j}}, \space j=1,\ldots,l'$, where $m_j \sim \text{Unif}([0,1]^3),\space \sigma_j \sim \text{Unif}([0,1])$\;
    \For{$\ell = 0,\ldots,L$}{
        Compute $\widehat{V}^{(\ell)}$ with columns given by the discretization of the functions $f_j, \space j=1,\ldots,l'$ on the Ansatz space for refinement level $\ell$\;
        Compute the SVD of the matrix $A_{\mathbf{0},0}^{(\ell)} = V_0^{(\ell)H} \Sigma_0^{(\ell)} W_0^{(\ell)}$\;
        \For{$i=1,\ldots, M^{(\ell)}$}{
            Draw a random sample $\bs{y}_i \sim \text{Unif}([-1,1]^N)$\;
            Compute $B_{\bs{y}_i}^{(\ell)}$ and $B_{\bs{y}_i}^{(\ell-1)}$ using $V_0^{(\ell)}$ and $W_0^{(\ell)}$, and $V_0^{(\ell-1)}$ and $W_0^{(\ell-1)}$, respectively\;
            Set $D_{\bs{y}_i}^{(\ell)} = B_{\bs{y}_i}^{(\ell)} - B_{\bs{y}_i}^{(\ell-1)}$\;
            %Compute $A_{\bs{y}_i, 0}^{(\ell)}$ and $A_{\bs{y}_i, 1}^{(\ell)}$\;
            %Set $\Sigma_{\bs{y}_i,0} = V_0 A_{\bs{y}_i,0} W_0\hm$\;
            %Set $B_{\bs{y}_i} = V_0\hm A_{\bs{y}_i,1} W_0 \Sigma_{\bs{y}_i,0}\inv$
        }
        Set $D^{(\ell)} = \frac{1}{M^{(\ell)}} \sum_{i=1}^{M^{(\ell)}} D_{\bs{y}_i}^{(\ell)}$\;
    }
    \Return{$B^{\text{ML}} = \sum_{\ell=1}^L D^{(\ell)}$}
    \caption{Multilevel Monte Carlo quadrature for the Dirichlet Laplacian eigenvalue problem}\label{alg:multilevel}
\end{algorithm}

%\comment{think if i want to include remarks}