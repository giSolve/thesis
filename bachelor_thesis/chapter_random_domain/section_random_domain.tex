\section{Modelling of random domains and surfaces}

%%CHANGE OMEGA 
In what follows, let $D_{\text{ref}} \subset \bR^3$ denote a Lipschitz domain with piecewise smooth surface $S_{\text{ref}} \coloneq \partial D_{\text{ref}}$ and $(\Omega, \mathcal{F}, \bP)$ be a complete probability space.
We assume the uncertainty in the obstacle to be encoded by a random domain field \comment{meter cita}.
Hence, we assume the existence of a uniform $\mathcal{C}^1$-diffeomorphism $\mathbf{\chi}_D \colon \closure{D}_{\text{ref}} \times \Omega \to \bR^3$, i.e.
\[
    \lVert \mathbf{\chi}_D (\omega)\rVert _{\mathcal{C}^1(\closure{D}_{\text{ref}};\bR^3)}, \lVert \mathbf{\chi}_D\inv (\omega)\rVert _{\mathcal{C}^1(\closure{D(\omega)};\bR^3)} \leq C  \quad \text{for } \bP\text{-a.e. } \omega \in \Omega
\]
such that
\[
    D(\omega) = \mathbf{\chi}_D (D_{\text{ref}}, \omega).
\]

meter más de lo de cómo pasar al parameter $y\in [-1,1]^N$.