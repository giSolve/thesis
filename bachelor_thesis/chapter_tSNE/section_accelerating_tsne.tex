\section{Accelerating t-SNE}

The original t-SNE algorithm is not very fast. Its runtime is $\mathcal{O}(n^2)$, which quickly becomes infeasible for datasets with $100,000$ or more points \textcolor{red}{check the exact number}. In this section, we will give an overview of the two most popular methods proposed to speed up the algorithm. 

\subsection{Barnes-Hut t-SNE}
This method uses tree-bases algorithms to speed up t-SNE and was proposed by van der Maaten in 2014 \cite{vdMaa14}. 

\begin{itemize}
    \item the gradient of the t-SNE loss function (KLD) has a natural interpretation as an $N$-body system in which all of the $N$ points in the low-dimensional embedding exert forces on each other. We then need to compute the resultant force on each of the points
\end{itemize}
We start by recalling the gradient of the t-SNE cost function: 
\begin{equation}
    \frac{\partial C}{\partial y_i} = 4 \sum_{j \neq i} (p_{ij} - q_{ij}) q_{ij} Z (y_i - y_j)
\end{equation}
with normalization term $Z = \sum_{k \neq l} (1+ \norm{y_k - y_l}^2 )^{-1}$. Notice that we can split the gradient into two parts 
\begin{equation}
    \frac{\partial C}{\partial y_i} = 4 (F_{\text{attr}} + F_{\text{rep}}) = 4 \left( \sum_{j \neq i } p_{ij} q_{ij} Z (y_i - y_j) - \sum_{j \neq i} q_{ij}^2 Z (y_i - y_j) \right) 
\end{equation}
where $F_{\text{attr}}$ denotes the sum of all attractive forces and $F_{\text{rep}}$ the sum of all repulsive forces. 

This next explanation is taken from \cite{LinStei22}. \textcolor{red}{TODO: maybe move explanation to the EE section, I think it would make sense there, or maybe do a whole section on the dynamical systems viewpoint}

Why does it make sense to call these attractive and repulsive forces? Since we want to minimize the cost function, we perform gradient descent and step in the direction of the negative gradient, so we consider the term
\begin{equation}
   - \frac{1}{4} \frac{\partial C}{\partial y_i} = \sum_{j \neq i } p_{ij} q_{ij} Z (y_j - y_i) - \sum_{j \neq i} q_{ij}^2 Z (y_j - y_i).  
\end{equation}
The first term is considered the attractive term, since it moves the point $y_i$ towards a weighted average of the other $y_i$. 
The weights $p_{ij} q_{ij} Z$ are bigger if the two points are close to each other (both in the low- and high-dimensional space). 
The second term has the opposite sign and thus pushes $y_i$ away from a weighted average of the other points. This time, however, the weights only depend on the closeness of points in the low-dimensional space. 
Put together, this means that the attractive term attracts points that are actually meant to be with each other (based on their similarity in the high-dimensional space) and the repulsive term pushes points apart that get too close in the embedding space, regardless of their real similarity. 

\subsubsection{Approximating Attractive Forces}
Computing the attractive force is not too expensive computationally, if we approximate input similarities and use vantage-point trees. Recall that input similarities $p_{ij}$ are computed based on a Gaussian kernel. 
Thus, $p_{ij}$ values corresponding to dissimilar input objects $x_i$ and $x_j$ are very small. 
So, it makes sense to develop a sparse approximation for the $p_{ij}$. Instead of computing $n^2$ pairwise similarities, we focus on the $\lfloor 3 \kappa \rfloor$ nearest neighbors of each of the $n$ input objects only, where $\kappa$ denotes the perplexity. We denote the nearest neighbor set of $x_i$ by $\mathcal{N}_i$. 

The similarities are thus given by 
\begin{equation}
    p_{j|i} = \begin{cases}
    \frac{\exp(-\norm{x_i - x_j}^2) / 2 \sigma_i^2}{\sum_{k \in \mathcal{N}_i} \exp(-\norm{x_i - x_k}^2) / 2 \sigma_i^2} & \text{ if } j \in \mathcal{N}_i \\
    0  & \text{ otherwise}
    \end{cases}
\end{equation}
which are again symmetrized then. 

We can find the nearest neighbor sets $\mathcal{N}_i$ in $\mathcal{O}(u n \log n)$ time by building a data structure called a vantage-point tree and performing nearest neighbor search with its help. For details, see \cite{vdMaa14}. 

\textcolor{red}{TODO: maybe say more about vantage-point trees here?}

\subsubsection{Approximating Repulsive Forces}
Naively computing the repulsive forces is not a good idea. It would be in $\mathcal{O}(n^2)$. 



\subsection{Fast Interpolation-Based t-SNE}